% research positions

\cventry{October 2011 -- May 2015}{PhD researcher}{Imperial College London}{PI: Dr Anil A. Bharath}{}{
During my PhD I have been involved in the following projects:
\begin{itemize}
\item Hippocampal models for localization: Proposed a biologically inspired method that uses neural networks to improve the visual recall of already visited places.
\item Appearance-based indoor localization from wearable cameras:
Developed a state-of-the-art pipeline for appearance-based localization
 in indoor spaces. This included the development of new descriptors based
 on filtering techniques that improved the performance on very
 ambiguous localization data, i.e. indoor corridors. %Appearance-based
 %methods are currently used in apps like \textit{Google Maps Live View}
 %used to improve localization and obtain information about the surroundings.
\item Sparse coding for localization: Adapted  successful techniques
in image compression and denoising to create ``dictionaries of places''
for learning a representation of different routes inside a building.
\item  The RSM dataset of ``visual paths'' for benchmarking visual localization algorithms. It contains more than 1.8 km of video sequences captured with mobile and wearable devices along 6 indoor locations. I developed a benchmark to evaluate different methods using C++ and MATLAB.
\item A house-hold products dataset (the SHORT-100 dataset). SHORT-100 contains more than 150,000 images capturing usage particularities of blind and partially sighted people.
\item Picture This... project. Developed an Android app and C++ (OpenCV) backend to provide user localization based on image-matching against a previoulsy acquired dataset.
\end{itemize}
}

\cventry{October 2008 -- February 2010}{Research Assistant}{The Minerva Project: Vodafone Spain, regional Government and University of Seville R\&D project}{PI: Dr Alejandro Carballar}{}{This initiative funded my final year project. I researched on the IMS protocols and worked on R\&D project monitoring and management. I was able to work in a multidisciplinary environment, studying the applications of my main research topic in other areas of interest such as Bioengineering or Social Services.
}
%\cventry{Fall 2008 -- Fall 2009}{Research assistant}{University of Delaware}{Dr. Keith Decker}{}{Designed and implemented a multi-agent system to use a coalition of electric vehicles as a distributed battery.
%The coalition communicates with PJM (a regional transmission organization) to participate in the frequency regulation market.
%}
%\cventry{Spring 2005 -- Spring 2008}{Research assistant}{University of Delaware}{Dr. Kathleen F. McCoy}{}{Fringe word prediction project.
%Primary researcher in adapting word prediction to the topic of discourse.
%Primary researcher in validating the link between keystroke savings (theoretical evaluation) and communication rate (real-world evaluation).}
%\cventry{Summer 2003}{CIS Fellowship}{University of Delaware}{Dr. Sandra Carberry}{}{SIGHT project (summarizing information graphics).
%Designed annotation guidelines for information graphics and created a collection of graphics with detailed syntactic and semantic information.
%Analyzed caption text for potential indicators of the intended communicative goal.}
